\documentclass[a4paper,UTF8]{article}
\usepackage{ctex}
\usepackage[margin=1.25in]{geometry}
\usepackage{color}
\usepackage{graphicx}
\usepackage{amssymb}
\usepackage{amsmath}
\usepackage{amsthm}
%\usepackage[thmmarks, amsmath, thref]{ntheorem}
\theoremstyle{definition}
\newtheorem*{solution}{Solution}
\newtheorem*{prove}{Proof}
\usepackage{multirow}

%--

%--
\begin{document}
\title{习题一}
\author{151250104, 卢以宁}
\maketitle


\section*{Problem 1}
若数据包含噪声,则假设空间中有可能不存在与所有训练样本都一致的假设,此时的版本空间是什么?在此情形下,试设计一种归纳偏好用于假设选择。

\begin{solution}
 此时的版本空间为空。选择的假设应符合尽量多的样本; 同时可遵循奥卡姆剃刀原则,通过正则项等方法避免过拟合。
\end{solution}

\section*{Problem 2}
对于有限样例,请证明
\[
\text{AUC} = \frac{1}{m^+m^-}\sum_{x^+\in D^+}\sum_{x^-\in D^-}\left(\mathbb{I}(f(x^+)>f(x^-))+\frac{1}{2}\mathbb{I}(f(x^+)=f(x^-))\right)
\]

\begin{prove}
由定义,\[ \text{AUC} =  \frac{1}{2} \sum^{m-1}_{i=1}(x_{i+1}-x_{i})(y_i+y_{i+1}) \] \\
而设排序后第k个样例为$s_k, (k = 2, 3, 4..., m)$, 对应的坐标为$(x_{k},y_{k})$,则
\begin{displaymath}
(x_{k}-x_{k-1})(y_k+y_{k-1}) = \left\{ \begin{array}{ll}
 0 & \textrm{ $s_k\in D^+$}\\
\frac{1}{m^-}\cdot y_k & \textrm{ $s_i\in D^-$}
  \end{array} \right.
\end{displaymath}
其中, $y_k$ 是由$s_k$及其之前的真正例数决定的. 
\[
y_k = \frac{1}{m^+}\cdot \sum_{x^+ \in D^+} \left( \mathbb{I} \left( f(x^+)>f(x_k) \right)  +  \frac{1}{2} {I}\left(f(x^+)=f(x_k)\right) \right)
\]
其中, $\frac{1}{2}$ 表示当某个真正例的预测值和$f(x_k)$相等时,它有一半的几率出现在$s_i$之前。
综上可得,
\[
\text{AUC} = \frac{1}{m^+m^-}\sum_{x^+\in D^+}\sum_{x^-\in D^-}\left(\mathbb{I}(f(x^+)>f(x^-))+\frac{1}{2}\mathbb{I}(f(x^+)=f(x^-))\right)
\]
\qed
\end{prove}

\section*{Problem 3} 
在某个西瓜分类任务的验证集中,共有10个示例,其中有3个类别标记为“1”,表示该示例是好瓜;有7个类别标记为“0”,表示该示例不是好瓜。由于学习方法能力有限,我们只能产生在验证集上精度(accuracy)为0.8的分类器。
\begin{itemize}
\item[(a)] 如果想要在验证集上得到最佳查准率(precision),该分类器应该作出何种预测?

此时的查全率(recall)和F1分别是多少?
\item[(b)] 如果想要在验证集上得到最佳查全率(recall),该分类器应该作出何种预测?

此时的查准率(precision)和F1分别是多少?
\end{itemize}
\begin{solution}
如下:
\begin{itemize}
\item[(a)] {因为准确率是0.8所以分错2个。由于有3个正例,至少有一个TP,标记它为正例,其他标记为反例,accuracy为1
$$ R =\frac{1}{3}, F_1 = \frac{2 \times P\times R }{P+R} = \frac{1}{2} $$ 

 }
\item[(b)] {将所有样例预测为正例, 查全率为100\%。此时\\

$$ P = \frac{TP}{TP+FP} = \frac{3}{10}, F_1 = \frac{2 \times P\times R }{P+R} = \frac{6}{13} $$ 
}
\end{itemize}

\end{solution}

\section*{Problem 4} 
在数据集$D_1,D_2,D_3,D_4,D_5$运行了$A,B,C,D,E$五种算法,算法比较序值表如表\ref{table:ranking}所示:
\begin{table}[h]
\centering
\caption{算法比较序值表} \vspace{2mm}\label{table:ranking}
\begin{tabular}{c|c c c c c}\hline
数据集 & 算法$A$ & 算法$B$  & 算法$C$  &算法$D$  &算法$E$ \\
\hline
$D_1$ & 2  & 3 &  1 &  5  & 4\\
$D_2$ & 5  & 4 &  2 &  3  & 1\\
$D_3$ & 4  & 5 &  1 &  2  & 3\\
$D_4$ & 2  & 3 &  1 &  5  & 4\\
$D_5$ & 3  & 4 &  1 &  5  & 2\\
\hline
平均序值 & 3.2 &  3.8 & 1.2 &  4 &  2.8 \\
\hline
\end{tabular}
\end{table}

使用Friedman检验$(\alpha=0.05)$判断这些算法是否性能都相同。若不相同,进行Nemenyi后续检验$(\alpha=0.05)$,并说明性能最好的算法与哪些算法有显著差别。
\begin{solution}
\[ 
\begin{split}
T_{\chi ^2} &= \frac{k-1}{k} \cdot \frac{12N}{k^2-1}\cdot \sum^{k}_{i=1}(r_i - \frac{k+1}{2})^2 \\
&= \frac{12N} {k(k+1)} \cdot \big ( \sum^k_{i=1} r_i^2 - \frac{k(k+1)^2}{4} \big ) \\
&= \frac{60} {5 \cdot 6} \cdot \big ( \sum^5_{i=1} r_i^2 - \frac{5 \cdot 36}{4} \big ) \\
&= 9.92 \\
T_{F} &= \frac{(N-1)T_{\chi ^2} }{N(k-1)-T_{\chi ^2} }  = 3.93 \\
\end{split}
\]
当 $\alpha = 0.05, N = 5, k = 5$时,F检验的临界值为 $3.007$, 故拒绝假设,认为算法性能不相同。然后使用 Nemenyi 后续检验 得到 \\
 $$ CD = q_{\alpha} \cdot \sqrt{\frac{k(k+1)}{6N}} = 2.728  \cdot \sqrt{\frac{5 \cdot 6}{6 \cdot 5}} = 2.728 $$
 经过比较得到,算法 C 与 算法 D 的性能显著不同,其余算法之间无显著不同。



\end{solution}
\end{document}