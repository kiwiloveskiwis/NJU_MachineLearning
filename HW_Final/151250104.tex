\documentclass[a4paper,UTF8]{article}
\usepackage{ctex}
\usepackage[margin=1.25in]{geometry}
\usepackage{color}
\usepackage{graphicx}
\usepackage{enumerate}
\usepackage{hyperref}
\usepackage{amsfonts, amsmath, amsthm, bm, amssymb}
\numberwithin{equation}{section}
\usepackage[ruled,vlined,lined,boxed,linesnumbered]{algorithm2e}
\theoremstyle{definition}
\newtheorem*{solution}{Solution}
\newtheorem*{prove}{Proof}
\newcommand{\indep}{\rotatebox[origin=c]{90}{$\models$}}
%--

%--
\begin{document}
\title{机器学习导论\\
综合能力测试}
\author{151250104, 卢以宁,  kiwiloveskiwis@gmail.com}
\maketitle
\section{[40pts] Exponential Families}
\label{Exponential Families}
指数分布族(\href{https://en.wikipedia.org/wiki/Exponential_family}{Exponential Families})是一类在机器学习和统计中非常常见的分布族, 具有良好的性质。在后文不引起歧义的情况下, 简称为指数族。

指数分布族是一组具有如下形式概率密度函数的分布族群:
\begin{equation}
f_X(x|\theta) = h(x) \exp \left(\eta(\theta) \cdot T(x) -A(\theta)\right)
\end{equation}  
其中, $\eta(\theta)$, $A(\theta)$以及函数$T(\cdot)$, $h(\cdot)$都是已知的。
\begin{enumerate}[(1)]
\item \textbf{[10pts]} 试证明多项分布(\href{https://en.wikipedia.org/wiki/Multinomial_distribution}{Multinomial distribution})属于指数分布族。

\item \textbf{[10pts]} 试证明多元高斯分布(\href{https://en.wikipedia.org/wiki/Multivariate_normal_distribution}{Multivariate Gaussian distribution})属于指数分布族。

\item \textbf{[20pts]} 考虑样本集$\mathcal{D}=\{ x_1,\cdots, x_n\}$是从某个已知的指数族分布中独立同分布地(i.i.d.)采样得到, 即对于$\forall i\in [1,n]$, 我们有$f( x_i|\boldsymbol\theta) = h(x_i) \exp \left ( {\boldsymbol\theta}^{\rm T}T(x_i) -A(\boldsymbol\theta)\right)$. 

对参数$\boldsymbol\theta$, 假设其服从如下先验分布:
\begin{equation}
p_\pi(\boldsymbol\theta|\boldsymbol\chi,\nu) = f(\boldsymbol\chi,\nu) \exp \left (\boldsymbol\theta^{\rm T} \boldsymbol\chi - \nu A(\boldsymbol\theta) \right )
\end{equation}
其中, $\boldsymbol\chi$和$\nu$是$\boldsymbol\theta$生成模型的参数。请计算其后验, 并证明后验与先验具有相同的形式。(\textbf{Hint}: 上述又称为“共轭”(\href{https://people.eecs.berkeley.edu/~jordan/courses/260-spring10/other-readings/chapter9.pdf}{Conjugacy}),在贝叶斯建模中经常用到)
\end{enumerate}

\begin{solution} 
(1) %%%%%%%%%%%%%%%%%%1.1
\begin{equation}
\begin{split}
f_X(x| \mathbf{p} ) &= \frac{n!} {x_1!\cdots x_k!} p_1^{x_1}\times\cdots\times p_k^{x_k}   \\
&= \frac{n!} {x_1!\cdots x_k!} \exp(\ln  p_1 \times {x_1}\times\cdots\times \ln p_k \times {x_k} )
\end{split}
\end{equation}  
其中 $n = \sum_ix_i $. 则令 $h(x) =  \frac{n!} {x_1!\cdots x_k!} $ , $\eta(\mathbf{p}) = \ln \mathbf{p}, T(\mathbf{x}) = \mathbf{x}, A(\mathbf{p}) = 0$ 即可。 


(2)  %%%%%%%%%%%%%%%%%% 1.2
\begin{equation}
\begin{split}
f_X(x | \Sigma, \mu ) &=\frac{\exp\left(-\frac 1 2 ({\mathbf x}-{\boldsymbol\mu})^\mathrm{T}{\boldsymbol\Sigma}^{-1}({\mathbf x}-{\boldsymbol\mu})\right)}{\sqrt{(2\pi)^k|\boldsymbol\Sigma|}}  \\
&= \frac{\exp\left(   { -\frac 1 2 (\mathbf{x^T}\Sigma^{-1}\mathbf{x}  -2\mathbf{x^T}\Sigma^{-1}\boldsymbol\mu + \boldsymbol\mu^T\Sigma^{-1}\boldsymbol\mu   )}     \right)}{\sqrt{(2\pi)^k|\boldsymbol\Sigma|}}  \\
&= \frac{\exp\left(   { \text{vec}(-\frac 1 2 \Sigma^{-1})^T \text{vec} (\mathbf{xx^T})  + \text{vec}(\Sigma^{-1}\boldsymbol\mu)^T\text{vec}(\mathbf{x^T}) -\frac 1 2 \boldsymbol{\mu^T\Sigma^{-1}\mu}   - \frac{1} {2} \ln |\boldsymbol\Sigma|)}     \right)}{\sqrt{(2\pi)^k}}  \\
\end{split}
\end{equation}  
则令 $h(x) =  \frac{1} {\sqrt{(2\pi)^k}}  $ , $\eta(\boldsymbol{\Sigma, \mu}) = \begin{bmatrix} -\frac 1 2 \boldsymbol\Sigma^{-1} \\  \boldsymbol{\Sigma^{-1}\mu} \end{bmatrix}        , T(\mathbf{x}) =\begin{bmatrix}  \mathbf{xx^T} \\  \mathbf{x^T}  \end{bmatrix}  , A(\Sigma, \mu ) = \frac 1 2 \boldsymbol{\mu^T\Sigma^{-1}\mu}   + \frac{1} {2} \ln |\boldsymbol\Sigma| $ 即可。 
\end{solution} 

(3) %%%%%%%%%%%%%%%%%%1.3
\begin{equation}
\label{a}
f( \mathbf{X} | \boldsymbol\theta) =( \prod_{i=1}^n h(x_i)) \exp \left ( {\boldsymbol\theta}^{\rm T} \sum_{i=1}^nT(x_i) -nA(\boldsymbol\theta)\right)
\end{equation}
\begin{equation}
\label{b}
p_\pi(\boldsymbol\theta|\boldsymbol\chi,\nu, \mathbf{X}) \propto ( \prod_{i=1}^n h(x_i)) f(\boldsymbol\chi,\nu) \exp \left (\boldsymbol\theta^{\rm T} (\boldsymbol\chi + \sum_{i=1}^nT(x_i)) - (\nu+n) A(\boldsymbol\theta) \right )
\end{equation}
从而, \ref{a} 与 \ref{b} 形式相同。

\newpage
\section{[40pts] Decision Boundary}
考虑二分类问题, 特征空间$X \in \mathcal{X}= \mathbb{R}^d$, 标记$Y \in \mathcal{Y}= \{0, 1\}$. 我们对模型做如下生成式假设:
\begin{itemize}
\item[-] attribute conditional independence assumption: 对已知类别, 假设所有属性相互独立, 即每个属性特征独立地对分类结果发生影响;
\item[-] Bernoulli prior on label: 假设标记满足Bernoulli分布先验, 并记$\Pr(Y=1) = \pi$. 
\end{itemize}

\begin{enumerate}[(1)]
\item \textbf{[20pts]} 假设$P(X_i | Y)$服从指数族分布, 即
\[
\Pr(X_i = x_i | Y = y) = h_i(x_i) \exp (\theta_{iy} \cdot T_i(x_i) - A_{i}(\theta_{iy}))
\]
请计算后验概率分布$\Pr(Y | X)$以及分类边界$\{x \in \mathcal{X}: P(Y=1 | X = x) = P(Y=0 | X =x)\}$. (\textbf{Hint}: 你可以使用sigmoid函数$\mathcal{S}(x)=1/(1+e^{-x})$进行化简最终的结果).

\item \textbf{[20pts]} 假设$P(X_i | Y=y)$服从高斯分布, 且记均值为$\mu_{iy}$以及方差为$\sigma_{i}^2$ (注意, 这里的方差与标记$Y$是独立的), 请证明分类边界与特征$X$是成线性的。 
\end{enumerate}
\begin{solution}
(1) %%%%%%%%%%%%%%%%%%%%%% 2.1.1
\begin{equation}
\Pr(\mathbf{X} | Y = y) = (\prod_{i=1}^d h_i(x_i)) \exp (\sum_{i=1}^d( \theta_{iy} \cdot T_i(x_i) - A_{i}(\theta_{iy})))
\end{equation}
\begin{equation}
\Pr(\mathbf{X}) = { (\prod_{i=1}^d h_i(x_i)) (\pi \exp (\sum_{i=1}^d( \theta_{i1} \cdot T_i(x_i) - A_{i}(\theta_{i1}))) +  (1-\pi) \exp (\sum_{i=1}^d( \theta_{i0} \cdot T_i(x_i) - A_{i}(\theta_{i0})))) }
\end{equation}

\begin{equation}
\begin{split}
\Pr(Y = 1 | \mathbf{X} ) &= \frac{\pi } { \left(\pi +  (1-\pi) \exp (\sum_{i=1}^d( (\theta_{i0} - \theta_{i1})\cdot T_i(x_i)  + A_{i}(\theta_{i1}) -A_{i}(\theta_{i0}))) \right) } \\
\Pr(Y = 0 | \mathbf{X} ) &= \frac{ 1 - \pi } { \left(1 - \pi +  \pi\cdot \exp (\sum_{i=1}^d( (\theta_{i1} - \theta_{i0})\cdot T_i(x_i)  + A_{i}(\theta_{i0}) -A_{i}(\theta_{i1}))) \right) }
\end{split}
\end{equation}
\end{solution}
分类边界:  %%%%%%%%%%%%%%%%%%%%%% 2.1.2
\begin{equation}
\begin{split}
&\pi ^ 2 \cdot \exp (\sum_{i=1}^d( (\theta_{i1} - \theta_{i0})\cdot T_i(x_i)  + A_{i}(\theta_{i0}) -A_{i}(\theta_{i1})))   \\
&= (1-\pi)^2 \exp (\sum_{i=1}^d( (\theta_{i0} - \theta_{i1})\cdot T_i(x_i)  + A_{i}(\theta_{i1}) -A_{i}(\theta_{i0})))
\end{split}
\end{equation}
化简可得: 
\begin{equation}
\begin{split}
\ln \frac{\pi}{1-\pi}  = \sum_{i=1}^d( (\theta_{i0} - \theta_{i1})\cdot T_i(x_i)  + A_{i}(\theta_{i1}) - A_{i}(\theta_{i0}))
\end{split}
\end{equation}
(2) %%%%%%%%%%%%%%%%%%%%%% 2.2
因为 $P(X_i | Y=y)$ 服从高斯分布, 所以: 
\begin{equation}
\begin{split}
h_i(x_i) &= \frac{1}{\sqrt{2\pi}} \\
 \theta_{iy} &= \begin{bmatrix} \frac{\mu_{iy}}{\sigma_i^2} \\ \frac{1}{-2\sigma_i^2} \end{bmatrix}     \\
 T_{i}(x_i) &= \begin{bmatrix} x_i \\ x_i^2 \end{bmatrix} \\
 A_{i}(\theta_{iy}) &=  \frac{\mu_{iy}^2}{2\sigma_i^2} +  \ln \sigma_i  \\
\end{split}
\end{equation}
从而分类边界为: 
\begin{equation}
\begin{split}
\ln \frac{\pi}{1-\pi}  = \sum_{i=1}^d( (\frac{\mu_{i0}- \mu_{i1}}{\sigma_i^2})x_i  + A_{i}(\theta_{i1}) - A_{i}(\theta_{i0}))
\end{split}
\end{equation}
可见与 $x_i $ 呈线性。

\newpage
\section{[70pts] Theoretical Analysis of $k$-means Algorithm} %%%%%%%%%%%%%%%%% Problem 3
给定样本集$\mathcal{D} = \{ \mathbf x_1,\mathbf x_2, \ldots, \mathbf x_n \}$, $k$-means聚类算法希望获得簇划分$\mathcal{C}=\{C_1,C_2,\cdots,C_k\}$, 使得最小化欧式距离
\begin{equation}
\label{eq-kmeans-l2}
J(\gamma, \mu_1,\ldots,\mu_k) = \sum_{i=1}^n \sum_{j=1}^k \gamma_{ij}||\mathbf x_i - \mu_j||^2
\end{equation} 
其中, $\mu_1, \ldots, \mu_k$为$k$个簇的中心(means), $\gamma \in \mathbb{R}^{n\times k}$为指示矩阵(indicator matrix)定义如下:若$\mathbf x_i$属于第$j$个簇, 则$\gamma_{ij} = 1$, 否则为0. 

则最经典的$k$-means聚类算法流程如算法\ref{algo:kmeans}中所示(与课本中描述稍有差别, 但实际上是等价的)。
\begin{algorithm}[] %%%%%%%%%%%%%%%%%%%% algorithm
\label{algo:kmeans}
\caption{$k$-means Algorithm}
\setcounter{AlgoLine}{0}
Initialize $\mu_1, \ldots, \mu_k$.\\
\Repeat{the objective function $J$ no longer changes}{
\textbf{Step 1}: Decide the class memberships of $\{\mathbf x_i\}_{i=1}^n$ by assigning each of them to its nearest cluster center.
\begin{align*}
\gamma_{ij} =
\begin{cases} 
1,& ||\mathbf x_i - \mu_j||^2 \le ||\mathbf x_i - \mu_{j'}||^2, \forall j' \\
0, & \text{otherwise} 
\end{cases}
\end{align*}\\
\textbf{Step 2}: For each $j \in \{1, \cdots, k\}$, recompute $\mu_j$ using the updated $\gamma$ to be the center of mass of all points in $C_j$: 
\begin{align*}
\mu_j = \frac{\sum_{i=1}^n \gamma_{ij}\mathbf x_i}{\sum_{i=1}^n \gamma_{ij}}
\end{align*}
}
\end{algorithm}

\begin{enumerate}[(1)]

\item \textbf{[10pts]} 试证明, 在算法\ref{algo:kmeans}中, \textbf{Step 1}和\textbf{Step 2}都会使目标函数$J$的值降低.

\item \textbf{[10pts]} 试证明, 算法\ref{algo:kmeans}会在有限步内停止。

\item {\textbf{[10pts]} 试证明, 目标函数$J$的最小值是关于$k$的非增函数, 其中$k$是聚类簇的数目。}

\item {\textbf{[20pts]} 记$\hat{\mathbf{x}}$为$n$个样本的中心点, 定义如下变量,
\begin{table}[h]
\centering
\label{table:equation}
\begin{tabular}{ l | c }
  \hline			
total deviation & $T(X) = \sum_{i=1}^n \lVert \mathbf x_i - \hat{\mathbf x}\rVert^2/n$ \\
intra-cluster deviation & $W_j(X) = \sum_{i=1}^n \gamma_{ij} \lVert\mathbf x_i - \mu_j \rVert^2/\sum_{i=1}^n \gamma_{ij}$ \\
inter-cluster deviation & $B(X) = \sum_{j=1}^k \frac{ \sum_{i=1}^n \gamma_{ij}}{n}  \lVert\mu_j -\hat{\mathbf x} \rVert^2$\\
  \hline  
\end{tabular}
\end{table}

试探究以上三个变量之间有什么样的等式关系?基于此, 请证明, $k$-means聚类算法可以认为是在最小化intra-cluster deviation的加权平均, 同时近似最大化inter-cluster deviation.}

\item { \textbf{[20pts]} 在公式\eqref{eq-kmeans-l2}中, 我们使用$\ell_2$-范数来度量距离(即欧式距离), 下面我们考虑使用$\ell_1$-范数来度量距离

\begin{equation}
\label{eq-kmeans-l1}
J'(\gamma, \mu_1,\ldots,\mu_k) = \sum_{i=1}^n \sum_{j=1}^k \gamma_{ij}||\mathbf x_i - \mu_j||_1
\end{equation}

\begin{itemize} 
\item \textbf{[10pts]} 请仿效算法\ref{algo:kmeans}($k$-means-$\ell_2$算法), 给出新的算法(命名为$k$-means-$\ell_1$算法)以优化公式\ref{eq-kmeans-l1}中的目标函数$J'$.
\item \textbf{[10pts]} 当样本集中存在少量异常点(\href{https://en.wikipedia.org/wiki/Outlier}{outliers})时, 上述的$k$-means-$\ell_2$和$k$-means-$\ell_1$算法, 我们应该采用哪种算法?即, 哪个算法具有更好的鲁棒性?请说明理由。
\end{itemize}}

\end{enumerate}

\begin{solution} %%%%%%%%%%%%%%%%%%%%%%%%%% 3.1
\begin{enumerate}[(1)]
\item   在 \textbf{Step 1} 中, $\forall i$, 令 \[ \hat j = \min_{1 \leq j \leq k} ||\mathbf x_i - \mu_j||^2 \]
又因为$\boldsymbol \gamma_i$是指示向量,\[ \sum_{j=1}^k \gamma_{ij}||\mathbf x_i - \mu_j||^2  \geq || \mathbf x_i - \mu_{\hat j}||^2 \]
故 $J(\gamma, \mu_1,\ldots,\mu_k) $若在第一步发生改变, 一定下降。  \\
 在 \textbf{Step 2} 中,  令 $X_{\in j} = \{\mathbf x_i | \gamma_{ij} = 1\}$代表在簇 $j$ 中点的集合, $n_j =|X_{\in j}|$ 为该集合大小, $\bar{ \mathbf x}$ 为该集合的均值。 则$\forall j , \forall \mathbf{a} $:
 \begin{equation}
 \begin{split}
 \sum_{x\in X_{\in j}}  ||\mathbf x - \mathbf{a} ||^2  &=   \sum_{x\in X_{\in j}}  (\mathbf{ x^Tx +  a^Ta - 2 x^Ta  })   \\
 &=  \sum_{x\in X_{\in j}}\mathbf{ x^Tx} +n_j \mathbf{a^Ta} - 2 n_j  {\bar{ \mathbf x}^T\mathbf a} 
 \end{split}
\end{equation}
而当 $\mathbf a =\bar{ \mathbf x}$时上式取得最小值。故 $J(\gamma, \mu_1,\ldots,\mu_k) $ 若在第二步发生改变,一定下降。
\item  %%%%%%%%%%%%%%%%%%%%%%%%%%%%%%%%%%%3.2
由于不同的 $J(\gamma, \mu_1,\ldots,\mu_k) $ 对应不同的$\boldsymbol \gamma$ ,  且每次更新时 $J$均下降(不下降时终止),  故$\boldsymbol \gamma$不会与先前重复。又由于 $\boldsymbol \gamma$ 最多有 $k^n$ 种可能取值,所以算法会在有限步终止。\\
\item  %%%%%%%%%%%%%%%%%%%%%%%%%%%%%%%%%% 3.3
令 $k = k_0$ 时取得 $J$ 的最小值的指示矩阵 $\gamma$不变,  $k = k_1$时将 $\mu_{k+1}$ 设为 $\mathcal{D}$ 中任意一个点, 则 $J$ 的值必不上升。从而$J$ 的最小值不上升。 \\
\item  %%%%%%%%%%%%%%%%%%%%%%%%%%%%%%%%%% 3.4
 令 $X_{\in j} = \{\mathbf x_i | \gamma_{ij} = 1\}$代表在簇 $j$ 中点的集合, $n_j =\sum_{i=1}^n \gamma_{ij}$ 为该集合大小, $\hat{ \mathbf x}$ 为该集合的均值. 则:
  \begin{equation}
 \begin{split}
  n_j W_j(X) + nB_j(X) &=  \sum_{x\in X_{\in j}} \lVert\mathbf x - \mu_j \rVert^2 + {n_j} \lVert\mu_j -\hat{\mathbf x} \rVert^2 \\
  &=  \sum_{x\in X_{\in j}}  {\mathbf{x^Tx}} - n_j ||\boldsymbol \mu_j||^2 +  {n_j}(||\boldsymbol \mu_j||^2 + ||\hat {\mathbf x} || ^ 2 - 2 \boldsymbol \mu_j^T\hat {\mathbf x }) \\
  &= \sum_{x\in X_{\in j}}  {\mathbf{x^Tx}} + n_j ||\hat {\mathbf x} || ^ 2 - 2 n_i \boldsymbol \mu_j^T\hat {\mathbf x } \\
  &= \sum_{x \in X_{\in j}} \lVert \mathbf x - \hat {\mathbf x}\rVert^2 
   \end{split} 
\end{equation}
从而:
  \begin{equation}
 \begin{split}
 \sum_{j=1}^k  \frac{ n_j}{n} W_j(X) + B(X)  &=   T(X)
    \end{split}
\end{equation}
由于算法迭代过程中$ T(X)$ 不变,而目标函数 $J$ 的值下降, 即  $ \sum_{j=1}^k  \frac{ n_j}{n} W_j(X) $ 的值下降, 所以 $B(X)$ 上升。所以可以认为“是在最小化intra-cluster deviation的加权平均, 同时近似最大化inter-cluster deviation”。
\item %%%%%%%%%%%%%%%%%%%%%%%%%%%%%%%%%% 3.5
设 d 代表维度。
\begin{algorithm}[] 
\label{algo:kmeans-l1}
\caption{$k$-means-$\ell_1$ Algorithm}
\setcounter{AlgoLine}{0}
Initialize $\mu_1, \ldots, \mu_k$.\\
\Repeat{the objective function $J$ no longer changes}{
\textbf{Step 1}: Decide the class memberships:
\begin{align*}
\gamma_{ij} =
\begin{cases} 
1,& ||\mathbf x_i - \mu_j||_1 \le ||\mathbf x_i - \mu_{j'}||_1, \forall j' \\
0, & \text{otherwise} 
\end{cases}
\end{align*}\\
\textbf{Step 2}: For each $j \in \{1, \cdots, k\}$, recompute $\mu_j$ using the updated $\gamma$ :
\begin{align*}
\forall d, \mu_j[d] = \text{median of } \{ \mathbf{x_i} | \gamma_ij = 1 \}
\end{align*}
}
\end{algorithm}
当样本集中存在少数异常点时, $k$-means-$\ell_1$算法具有更好的鲁棒性。因为采用平均时,与这些异常点最近的簇的中心点 很可能受到较大影响从而偏离本应在的中心, 而采用中位数时,该中心点不会发生太大变化。
\end{enumerate}
\end{solution}


\newpage %%%%%%%%%%%%%%%%%%%%%%%%%%%%%%% Problem 4
\section{[50pts] Kernel, Optimization and Learning}
给定样本集$\mathcal{D} = \{(\mathbf{x}_1,y_1),(\mathbf{x}_2,y_2),\cdots,(\mathbf{x}_m,y_m)\}$, $\mathcal{F} = \{\Phi_1 \cdots,\bm \Phi_d\}$为非线性映射族。考虑如下的优化问题
\begin{equation}
\label{eq-primal}
\min_{\mathbf w, \mu\in \Delta_q} \quad \frac{1}{2} \sum_{k=1}^d \frac{1}{\mu_k}\lVert\mathbf w_k\rVert_2^2 + C\sum_{i=1}^m \max\left\lbrace 0,1 - y_i\left(\sum_{k=1}^d \mathbf w_k \cdot \bm \Phi_k(\mathbf{x}_i)\right) \right\rbrace
\end{equation}
其中, $\Delta_q = \left\lbrace \bm{ \mu} | \mu_k\geq 0, k=1,\cdots,d; \lVert \bm{ \mu} \rVert_q = 1\right\rbrace$.

\begin{enumerate}[(1)]
\item{ \textbf{[30pts]} 请证明, 下面的问题\ref{eq-dual}是优化问题\ref{eq-primal}的对偶问题。
\begin{equation}
\label{eq-dual}
	\begin{split}
\max_{\bm \alpha} &\quad 2\bm \alpha^\mathrm T \mathbf{1}- \left\lVert
 \begin{matrix}
   \bm \alpha^\mathrm{T}\mathbf Y^\mathrm{T} \mathbf K_1 \mathbf Y  \bm \alpha \\
   \vdots \\
  \bm \alpha^\mathrm{T}\mathbf Y^\mathrm{T} \mathbf K_d \mathbf Y  \bm \alpha 
  \end{matrix}
  \right\rVert_p\\
  \text{s.t.} &\quad  \mathbf{0} \leq \bm \alpha  \leq \mathbf{C} 
  \end{split}
\end{equation}
其中, $p$和$q$满足共轭关系, 即$\frac{1}{p}+\frac{1}{q}=1$. 同时, $\mathbf Y = \text{diag}([y_1,\cdots,y_m])$, $\mathbf K_k$是由$\bm \Phi_k$定义的核函数(kernel).}
\item{ \textbf{[20pts]} 考虑在优化问题\ref{eq-dual}中, 当$p=1$时, 试化简该问题。}
\end{enumerate}

\begin{solution}
一个偷懒的人在此悄悄回望.
~\\
~\\
~\\
\end{solution}
\end{document}